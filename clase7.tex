\documentclass[12pt]{article}
\usepackage[utf8]{inputenc}
\usepackage[spanish]{babel}
\usepackage[top=1in,bottom=1in,left=1in,right=1in]{geometry}

%paquetes de matematica
\usepackage{amsmath}
\usepackage{amssymb}
\usepackage[mathscr]{euscript}


%comandos de secciones 
\newcommand{\teorema}{\section{Teorema}}
\newcommand{\definicion}{\section{Definición}}
\newcommand{\ejercicios}{\section{Ejercicios}}
\newcommand{\solucion}{\section{Solución}}

%comandos varios utiles (agregar es posible)
\newcommand{\Rn}[1]{\mathbb{R}^{#1}}
\newcommand{\vect}[1]{\textbf{#1}}
\newcommand{\vecti}[2]{\textbf{#1}_{#2}}
\newcommand{\bola}[2]{\mathcal{B}_{#1}(#2)}
\newcommand{\parcial}[2]{\frac{\partial #1}{\partial #2}}

\title{Clase \# 7 de Análisis 3}
\author{Equipo clases a \LaTeX}

\begin{document}
	
	\maketitle
	\tableofcontents
	
	\teorema 
	\underline{Condición suficiente de diferenciabilidad} \\
	
	Sea  $f \, : \, S \subset \Rn{n} \mapsto \Rn{}$, $S$ abierto, $\vec{x} \in S$. $ D_1 f, \cdots, D_n f $ existen y son continuas en $B(\vec{x}, r) \subset S$. Entonces $f$ es diferenciable en $\vec{x}$

	\teorema
	\underline{Regla de la cadena} \\
	
	Consideremos $f \, : \, S \subset \Rn{n} \mapsto \Rn{}$, $S$ abierto, $\vec{r} \, : \, I \subset \Rn{} \mapsto S$, y $g(t) = f(\vec{r}(t))$, $t \in I$. Sea $t \in I$ donde $\vec{r} \, ^{\prime} (t)$ existe y supongamos que $f$ es diferenciable en $\vec{r}(t)$, entonces existe $g^{\prime} (t)$ y tenemos que 
	
	$$ g ^{\prime} (t) = \nabla  f(\vec{x}) \cdot \vec{r}\,^{\prime} (t)$$
	
	donde $\vec{x} = \vec{r} (t)$.
	
    \pagebreak
    

	\ejercicios
	
	\begin{enumerate}
		\item \textit{Halle  el vector gradiente si}
		
		\begin{enumerate}
		    \item $f(x,y) \, = \, x^{2} \, + \, y^{2} \sin(xy).$
		    \item $f(x,y,z) \, = \, x^{2} \, - \, y^{2} \, +2z^{2}.$
		\end{enumerate}
		
		\item \textit{Calcule la derivada direccional de $f(x,y,z) \, = \, x^{2} \, + \, 2y^{2} \, + \, 3z^{2}$ en $(1,1,0)$ en la dirección de $\vec{e}_{1} - \vec{e}_{2} + 2\vec{e}_{3}$.}
		
		\item \textit{Hallar los puntos (x,y) y las direcciones para las que la derivada direccional de $f(x,y) \, = \, 3x^{2} \, + \, y^{2}$ tiene valor máximo, si (x,y) pertenece a la circunferencia $x^{2} \, + \, y^{2} \, = \, 1$.}
		
		\item \textit{Supóngase que f es diferenciable en cada punto de $B(\vec{x}, r)$. Demuestre:}
		
		\begin{enumerate}
		    \item Si $\nabla f(\vec{y}) = \vec{0}$ para todo $\vec{y} \in B(\vec{x}, r)$ entonces f es constante en $B(\vec{x}, r)$.
		    \item Si $f(\vec{y}) \leqslant f(\vec{x})$ para todo $\vec{y} \in B(\vec{x}, r)$ entonces $\nabla f(\vec{x}) = \vec{0}.$
		\end{enumerate}
		
		\item \textit{Hallar la derivada direccional de $f(x,y) \, = \, x^{2} \, - \, x \, + \, 2$ a lo largo de $y \, = \, x^{2} \, - \, x \, + \, 2$ en el punto (1,2). Use regla de la cadena.}
		
		\item \textit{Sea f un campo escalar no constante diferenciable en todo el plano y c una constante. Supongamos que la ecuación $f(x,y) \, = \, c$ describe una curva $\mathscr{C}$ que tiene tangente en cada uno de sus puntos. Demuestre que f tiene las siguientes propiedades en cada punto de $\mathscr{C}$}
		
		\begin{enumerate}
		    \item $\nabla f$ es un vector normal a $\mathscr{C}$.
		    \item La derivada direccional de f a lo largo de $\mathscr{C}$ es cero.
		    \item La derivada direccional de f tiene su valor máximo en la dirección del vector normal a $\mathscr{C}$.
		\end{enumerate}
		
		\item \textit{Sea $f \, : \, S \subset \Rn{3} \mapsto \Rn{}$, $S$ es abierto, f es diferenciable en $S$. Sea c una constante y consideremos la superficie de nivel $\mathscr{H} \, = \, \{\vec{y} \in S \, ; \, f(\vec{y}) \, = \, c \}$. Sea $\vec{a} \in \mathscr{H}$. Demuestre que la ecuación del plano tangente a la superficie $\mathscr{H}$ satisface la ecuación}
		
		$$ \nabla f (\vec{a}) \cdot (\vec{x} - \vec{a}) \, = \, 0 $$
		
		\item \textit{Sea $f(x,y) \, = \, \sqrt{| xy |}$. Compruebe que $\dfrac{\partial f}{\partial x} \, = \, \dfrac{\partial f}{\partial y} \, = \, 0$ en (0,0) ¿Tiene la superficie $z \, = \, f(x,y)$ plano tangente en (0,0)?}
		
	\end{enumerate}
	
	\pagebreak 
	
	\solucion
	
	\begin{enumerate}
	    \item 
	    \begin{enumerate}
	        \item Sea $f \,:\,  \Rn{2} \mapsto \Rn{}$ tal que $f(x,y) = x^{2} \, + \, y^{2} \sin(xy)$. Como \linebreak $ \dfrac{\partial f }{\partial x} (x,y) = 2x + y^{3} cos(xy)$ y $\dfrac{\partial f}{\partial y} (x,y) = 2y sen(xy) + y^{2} x cos(xy)$, tenemos que 
	    
	    \begin{eqnarray*}
	        \nabla f (x,y) & = & \dfrac{\partial f }{\partial x} i + \dfrac{\partial f}{\partial y} j \\[0.2 cm]
	        & = & (\,2x + y^{3} cos(xy)\,) \, i + (\,2y sen(xy) + y^{2} x cos(xy)\,) \, j
	    \end{eqnarray*}
	  $\hfill \square$
	  
	  \item Sea $f \, :\, \Rn{3} \mapsto \Rn{}$ tal que $f(x,y,z) =\, x^{2} \, - \, y^{2} \, +2z^{2}$. Entonces
	  
	  \begin{eqnarray*}
	  \nabla f(x,y,z) & = & \dfrac{\partial f }{\partial x} i + \dfrac{\partial f}{\partial y} j + \dfrac{\partial f}{\partial y} k \\[0.2 cm]
	  & = & 2x \,i \, - \, 2y \, j \, + \, 4z \, k
	  \end{eqnarray*}
	  $\hfill \square$
	    \end{enumerate}
	  
	  \item Sea $f \, : \, \Rn{3} \mapsto \Rn{}$ la función definida por $f(x,y,z) \, = \, x^{2} \, + \, 2y^{2} \, + \, 3z^{2} $, consideremos el punto P(1,1,0) y el vector $\vec{v} = e_{1} - e_{2} + 2e_{3} = (1, -1, 2)$. Entonces tenemos que el vector unitario $\vec{u}$ en dirección del vector $\vec{v}$ es 
	  
	  $$ \vec{u} \, = \, \dfrac{1}{|| \vec{v}||} \vec{v}\, = \, \Bigl( \dfrac{\sqrt{6}}{6}, \dfrac{- \sqrt{6}}{6}, \dfrac{\sqrt{6}}{3} \Bigr) $$
	  
	  Ahora, como $D_{\vec{v}}(x,y,z) \, = \, \nabla f(x,y,z) \cdot \vec{u}$ tenemos que 
	  
	  $$  D_{\vec{v}}(x,y,z) \, = \, \dfrac{\sqrt{6}}{6} x \, - \, \dfrac{- 2\sqrt{6}}{3} y \, + \, 2 \sqrt{6}z.$$
	  
	  Por tanto,
	  
	  $$ D_{\vec{v}}(1,1,0) \, = \,  \dfrac{\sqrt{6}}{6} - \dfrac{- 2\sqrt{6}}{3} \, = \, - \dfrac{\sqrt{6}}{3} $$
	  
	  $\hfill \square$
	  
	  \pagebreak
	  
	  \item 
	  
	  \item 
	  \begin{enumerate}
	      \item Sea $\vec{y} \in B(\vec{x}, r)$, consideremos la función $g:[0,1] \mapsto \Rn{}$ definida por \linebreak $g(t) = f(\, (1-t)\vec{x} + t \vec{y} \, )$. Como f es diferenciable en el conjunto convexo $B(\vec{x}, r)$, tenemos que g es continua en [0,1] y diferenciable en (0, 1). \\
	      
	      Si aplicamos que teorema del valor medio para funciones de un variable, tendremos que 
	      
	      $$g(1) - g(0) \, = \, g^{\prime}(\theta), \;\; \theta \in (0,1)$$
	      
	      Como $g(1) = f(\vec{y})$ y $g(0) = f(\vec{x})$, tenemos que
	      
	      $$f(\vec{y}) - f(\vec{x}) = g^{\prime}(\theta)$$
	      
	      Ahora, si aplicamos la regla de la cadena a $g(\theta) =  f(\, (1-\theta)\vec{x} + \theta \vec{y} \, ) $ tendremos que
	      
	      $$ g^{\prime}(\theta) \, = \, \nabla f(\vec{r}(\theta)) \cdot (\vec{y} - \vec{x})$$
	      
	      donde $\vec{r}(\theta) = \, (1-\theta)\vec{x} + \theta \vec{y} $. Por lo que 
	      
	      $$f(\vec{y}) - f(\vec{x}) \, = \, \nabla f(\vec{r}(\theta)) \cdot (\vec{y} - \vec{x})$$
	      
	      Como $ \vec{r}(\theta) \in B(\vec{x}, r)$ para toda $\theta \in (0,1)$, tenemos que $\nabla f(\vec{r}(\theta)) = \vec{0}$, y en consecuencia
	      
	      $$ f(\vec{y}) - f(\vec{x}) \, = \, \vec{0} \cdot (\vec{y} - \vec{x}) \; \Longrightarrow \; f(\vec{y}) \, = \,  f(\vec{x}) $$
	      
	      Por tanto, para todo $\vec{y} \in B(\vec{x}, r)$ f es constante. $\hfill \square$
	      
	      
	      
	      
	      
	  \end{enumerate}
	   
	   
	   
	   
	\end{enumerate}
	
\end{document}