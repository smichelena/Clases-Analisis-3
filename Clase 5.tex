\documentclass[12pt]{article}
\usepackage[utf8]{inputenc}
\usepackage[spanish]{babel}
\usepackage[top=1in,bottom=1in,left=1in,right=1in]{geometry}

%paquetes de matematica
\usepackage{amsmath}
\usepackage{amssymb}

%comandos de secciones 
\newcommand{\teorema}{\section{Teorema}}
\newcommand{\definicion}{\section{Definicion}}
\newcommand{\ejercicios}{\section{Ejercicios}}

%comandos varios utiles (agregar es posible)
\newcommand{\Rn}[1]{\mathbb{R}^{#1}}
\newcommand{\vect}[1]{\textbf{#1}}
\newcommand{\vecti}[2]{\textbf{#1}_{#2}}
\newcommand{\bola}[2]{\mathcal{B}_{#1}(#2)}

\title{Clase \# 5 de Análisis 3}
\author{Equipo clases a \LaTeX}

\begin{document}
	
	\maketitle
	
	
	\definicion \textbf{Derivadas en funciones de varias variables:}
	Sea $f: S\subset \mathbb{R^n}\rightarrow \mathbb{R}$, $S$ conjunto abierto, $\vec{x}_\bullet \in S$, $\vec{u}\in \mathbb{R^n}$, $||\vec{u}||=1$, Supongamos que existe el límite 
	$$D_{\vec{u}}f(\vec{x}_\bullet)=\lim_{h\rightarrow \infty}\frac{f(\vec{x}_\bullet+h\vec{u}-f(\vec{x}_\bullet)}{h}$$
	llamaremos a $D_{\vec{u}}f(\vec{x}_\bullet)$ derivada direccional de $f$ en $\vec{x}_\bullet$, según dirección $||\vec{u}||$.
	\definicion
	Cuando $\vec{u}=e_1=(1,0,...,0)$ se escribe:
	$$D_{\vec{u}}f(\vec{x}_\bullet)=\frac{\partial f}{\partial x_1}(\vec{x}_\bullet)=D_1f(\vec{x}_\bullet)$$
	derivada parcial de $f$ con respecto a $x_1$, $\vec{x}=(x_1,x_2,...,x_n)$ evaluada en $\vec{x}_\bullet$.
	
	Análogamente:
	$$\frac{\partial f}{\partial x_k}(\vec{x}_\bullet)=D_kf(\vec{x}_\bullet)$$

	
\textbf{Observaciones}\\

La existencia de derivadas parciales no implica continuidad:\\

\textbf{Ejemplo}
$$f(x,y)=
\begin{cases}
x+y \quad \text{ si } x=0 \text{ ó } y=0\\
1  \quad \text{ cualquier otro caso }
\end{cases}$$

$D_1(0,0)=D_2f(0,0)=1$ pero $f$ no es contínua en  $(0,0)$.
\end{document}
