\documentclass[12pt]{article}
\usepackage[utf8]{inputenc}
\usepackage[spanish]{babel}
\usepackage[top=1in,bottom=1in,left=1in,right=1in]{geometry}

%paquetes de matematica
\usepackage{amsmath}
\usepackage{amssymb}
\usepackage[mathscr]{euscript}


%comandos de secciones 
\newcommand{\teorema}{\section{Teorema}}
\newcommand{\definicion}{\section{Definición}}
\newcommand{\ejercicios}{\section{Ejercicios}}
\newcommand{\solucion}{\section{Solución}}
\newcommand{\ejemplo}{\section{Ejemplo}}

%comandos varios utiles (agregar es posible)
\newcommand{\Rn}[1]{\mathbb{R}^{#1}}
\newcommand{\vect}[1]{\textbf{#1}}
\newcommand{\vecti}[2]{\textbf{#1}_{#2}}
\newcommand{\bola}[2]{\mathcal{B}_{#1}(#2)}
\newcommand{\parcial}[2]{\frac{\partial #1}{\partial #2}}

\title{Clase \# 9 de Análisis 3}
\author{Equipo clases a \LaTeX}

\begin{document}
	
	\maketitle
	
	\definicion
	
	Sea $f \, : \, S \subset \Rn{n} \mapsto \Rn{}$ y $\vec{a} \in S$. Diremos que \underline{f tiene un máximo absoluto en $\vec{a}$} si \\
	\begin{equation}
	    f(\vec{x})  \leqslant f(\vec{a})
	\end{equation} \\

	Por otra parte, decimos que \underline{f tiene un máximo relativo en $\vec{a}$} si existe $B(\vec{a}, r) \subset S$ tal que $\vec{a}$ es un máximo absoluto en $B(\vec{a}, r)$. Llamamos a $f(\vec{a}) $ máximo relativo de f en S.
	
	\paragraph{Nota:} se deduce de manera análoga la definición de mínimo absoluto y mínimo relativo de f. Basta con hacer $ f(\vec{a}) \leqslant  f(\vec{x})$ en (1).
	
	\definicion
	
	Sea $f \, : \, S \subset \Rn{n} \mapsto \Rn{}$, $S$ abierto, $\vec{a} \in S$, y f diferenciable en $\vec{a}$. Si $\nabla f(\vec{a}) = \vec{0}$ entonces diremos que \underline{$\vec{a}$ es un punto estacionario de f}. \\
	
	Si para todo $B(\vec{a}, r)$ existen puntos $\vec{x}$ para los cuales $f(\vec{x}) < f(\vec{a}) $ y otros puntos $\vec{y}$ para los cuales $f(\vec{x}) > f(\vec{a})$ diremos que \underline{$\vec{a}$ es un punto de ensilladura.}
	
	\definicion
	Sea $f \, : \, S \subset \Rn{n} \mapsto \Rn{}$, $S$ abierto, f con derivadas parciales de segundo orden \linebreak $D_{ij} f = D_i[ D_j f]$. Llamamos a $\mathscr{H}(\vec{x}) \, = \, [D_{ij} f(\vec{x})]_{i,j = 1} ^{n}$ \underline{matriz hessiana} de f en $\vec{x}$.
	
	\pagebreak
	
	\teorema
	\underline{Fórmula de Taylor de segundo orden para campos escalares.} \\
	
	Sea $f \, : \, S \subset \Rn{n} \mapsto \Rn{}$, $S$ abierto, $\vec{a} \in S$, f con derivadas parciales segundas \underline{continuas} en $B(\vec{a}, r) \subset S$. Entonces para todo $\vec{y} \in \Rn{n}$ tal que $\vec{a} + \vec{y} \in B(\vec{a}, r)$ se cumple que:
	
	$$ f(\vec{a} + \vec{y}) \, = \, f(\vec{a}) + \nabla f(\vec{a}) \cdot \vec{y} + \dfrac{1}{2!} \vec{y} \,  \mathscr{H}(\vec{a})\, \vec{y} \, ^{t} + || \vec{y} || E_{2}(\vec{a}, \vec{y})$$
	
	donde $E_{2}(\vec{a}, \vec{y}) \rightarrow 0$ cuando $\vec{y} \rightarrow 0$.
	
	\paragraph{Nota:} recordemos que $\vec{y} \, \mathscr{H}(\vec{a})\,\vec{y} \, ^{t}$ es una forma cuadrática
	
	
	\definicion
	
	\underline{Definición de valores característicos y vectores característicos} \\
	
	Sea $ A \in M_{n \times n} (\mathbb{R}).$ Si $\lambda \in \mathbb{C}$ y $v \in \mathbb{C}^{n}  \setminus \{ \textbf{0} \}$ satisface 
	
	\begin{equation}
	    A v = \lambda v
    \end{equation}
    
    entonces $\lambda$ se denomina $\textbf{valor característico}$ de $A$ y el vector v se denomina $\textbf{vector característico}$ de $A$ correspondiente a $\lambda$.
    
    \paragraph{Nota:} Los valores y vectores característicos también se denominan autovalores y autovectores o eigenvalores y eigenvectores.
    
    \pagebreak
    
    \ejemplo
    
    Sea la matriz $A \, = \, 
    \begin{pmatrix}
            3 & 2 \\
            -5 & 1 \\
            \end{pmatrix} $ y el vector $\vec{v}_{1} \, = \, \begin{pmatrix}
            1 \, + \, 3i \\
            -5  \\
        \end{pmatrix} $, entonces
    \begin{eqnarray*}
    A \vec{v}_{1} & = & \begin{pmatrix}
            3 & 2 \\
            -5 & 1 \\
            \end{pmatrix} \begin{pmatrix}
            1 \, + \, 3i \\
            -5  \\
            \end{pmatrix} \\[0.2 cm]
    & = & \begin{pmatrix}
            -7 \, + \, 9i \\
            -10 \, - \, 15i\\
            \end{pmatrix} \\[0.2 cm]
    & = & (2 \, + \, 3i) \begin{pmatrix}
            1 \, + \, 3i \\
            -5  \\
            \end{pmatrix}
    \end{eqnarray*}
    
En virtud de la definición anterior, tenemos que $\lambda _{1} = 2 \, + \, 3i $ es un valor característico de $A$ y $\vec{v}_{1}$ es un vector característico de $A$ correspondiente a $\lambda _{1} .$ 

De manera analoga, si $\vec{v}_{2} \, = \, \begin{pmatrix}
            1 \, -\, 3i \\
            -5  \\
            \end{pmatrix} $, entonces 
            
\begin{eqnarray*}
A \vec{v}_{2}& = & \begin{pmatrix}
            3 & 2 \\
            -5 & 1 \\
            \end{pmatrix} \begin{pmatrix}
            1 \, - \, 3i \\
            -5  \\
            \end{pmatrix} \\[0.2 cm]
& = & \begin{pmatrix}
            -7 \, - \, 9i \\
            -10 \, + \, 15i\\
            \end{pmatrix} \\[0.2 cm]
& = & (2 \, - \, 3i) \begin{pmatrix}
            1 \, - \, 3i \\
            -5  \\
            \end{pmatrix}
\end{eqnarray*}

Por lo que $\lambda _{2} = 2 \, - \, 3i $ también es un valor característico de $A$ y $\vec{v}_{2} = \begin{pmatrix}
            1 \, - \, 3i \\
            -5  \\
            \end{pmatrix}$ es un vector característico de $A$ correspondiente a  $\lambda_{2}.$
	
    
    \definicion
    
    Sea $ A \in M_{n \times n} (\mathbb{R})$ y $\mathscr{I} $ la matriz identidad $n \times n$, decimos que 

    \begin{equation}
        p(\lambda) = det| A \, - \, \lambda \mathscr{I}|
    \end{equation}
    es el $\textbf{polinomio característico}$ de la matriz $A$. \\ 
    
    La ecuación (3) se denomina $\textbf{ecuación característica}$ de $A$.
    
    \pagebreak
    
    \ejemplo
    
    Consideremos nuevamente la matriz $A \, = \, \begin{pmatrix}
            3 & 2 \\
            -5 & 1 \\
            \end{pmatrix} $ y el polinomio característico de $A$. Entonces 
            
    \begin{eqnarray*}
        p(\lambda) & = & det  \begin{pmatrix}
            3 - \lambda & 2 \\
            -5 & 1 -\lambda \\
            \end{pmatrix}  \\[0.2 cm]
                & = & \lambda ^{2} -4 \lambda +13
    \end{eqnarray*}

Ahora si $p(\lambda) = 0$ entonces $\lambda ^{2} -4 \lambda +13 = 0$, lo que implica que $\lambda = 2 \pm 3i$. Por tanto, $\lambda_{1} = 2 + 3i$ y $\lambda_{2} = 2 - 3i$ son valores característicos de la matriz $A$.
	
	\teorema
	Sea $A \, = [a_{ij}]$ una matriz simétrica $n \times n$ consideremos 
	
	$$ Q(\vec{y}) \, = \, \vec{y} A \vec{y} \, ^{t} \, = \, \sum _{i = 1} ^{n} \sum _{j = 1} ^{n} a_{ij} y_{i} y_{j}$$
	
	Entonces:
	
	\begin{enumerate}
	    \item $Q(\vec{a}) > 0$ para todo $ \vec{y} \neq 0$ si y solo si todos los autovalores de A son positivos.
	    
	    \item  $Q(\vec{a}) < 0$ para todo $ \vec{y} \neq 0$ si y solo si todos los autovalores de A son negativos.
	\end{enumerate}
	
	
	\teorema

	Sea $f \, : \, S \subset \Rn{n} \mapsto \Rn{}$, $S$ abierto, $\vec{a} \in S$ tal que $\nabla f(\vec{a}) = \vec{0}$, f con derivadas parciales segundas \underline{continuas} en $B(\vec{a}, r) \subset S$, $\mathscr{H} (\vec{a})$ la matriz hessiana de f en $\vec{a}.$ Entonces:
	
	\begin{enumerate}
	    \item Si todos los autovalores de $\mathscr{H} (\vec{a})$ son positivos, f tiene un mínimo relativo en $\vec{a}$.
	    
	    \item Si todos los autovalores de $\mathscr{H} (\vec{a})$ son negativos, f tiene un máximo relativo en $\vec{a}$.
	    
	    \item Si $\mathscr{H} (\vec{a})$ tiene autovalores positivos y negativos, entonces f tiene un punto ensilladura en $\vec{a}$.
	    
	\end{enumerate}
	
	
	
	
	

	
    
    
    
	
	

\end{document}
